\chapter{Entropy Approach}
\textit{by Florian Fallenbüchel}\\
This chapter will explain our approach on the classification of the
genre via the entropy and fractal dimension of the song. This approach
is interesting, because it does not consider musical information such as
harmony, melody, beat and tempo, heavily reducing the size of the data
to be processed during training. We follow the work of Goulart
\textit{et al.}~\cite{entropy} from 2012, who used the entropy approach
together with combined support vector machines. They got 100\%
classification accuracy on a set of 90 songs equally distributed over 3
different genres, with 80-90\% of the data used for training. We try
to extend this method to a larger amount of classes, combining it with
neural networks for classification.\\
For the calculation of the entropy, we need the energy approach of
signal theory, as presented in Elements of Information
Theory~\cite{infotheo} by Cover and Thomas. The song is divided into
frames of 1024 samples with 50\% overlap between consecutive frames.
The entropy (E) then is defined as
\begin{equation}
	E = - \sum_{i=0}^{1023} p_{i} \log_{2}(p_{i}),
\end{equation}
with $p_{i}$ being the proportion of the


\begin{center}
\begin{tabular}{ p{\textwidth /4}  p{\textwidth *3 /5 }}

	Description & The way developers approach a task can provide detailed information about the structure of the project they are working on.
	Using this knowledge, other team-members can familiarize themselves with new parts of the project faster.
	This information can be displayed as trace links, which show the source code entities, that are relevant to a particular requirement specification.
	While the users can benefit from these trace links, their creation should not interfere with the normal workflow. Why is this even still on my hard-drive, I handed it in years ago.\\
	\hline
	User Subtasks & User Subtask 1.1: perform interactions\\
	& User Subtask 1.2: commit changes related to an issue\\
	& User Subtask 1.3: browse linked code entities\\	
\end{tabular}
	\label{tab:somerandomtablethatshouldnotbehere}
\captionof{table}{If you don't add a caption, the table wont show up in the list of tables. I invested a lot of clicks into making that, so you better think of something clever to put there.}
\end{center}

Notice how this tabular doesn't float all over the place? use regular tables if you prefer it that way.