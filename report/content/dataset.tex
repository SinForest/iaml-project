\chapter{Dataset}\label{dataset}
\textit{by Florian Fallenbüchel}\\
The dataset which we will use in our experiments consists of 106574
songs, downloaded from the Free Music Archive~\cite{fma_dataset}. This
gives us 917 GB of Creative Commons-licensed audiofiles, tagged with the
respective genres. While a song may have multiple subgenres from a range
of 161 genres, the data set assigns a single top genre to each song. We
will use these top genres as labels in our experiments. There are 16
different top genres in the data, while some songs have tagged an empty
string.

\begin{figure}[!htb]
	\centering
	\includegraphics[width=1.0\textwidth]{images/genredist.png}
	\caption{Distribution of the genres over the dataset.
	About 53$\%$ of the data (left bar) has no top genre tagged.}
	\label{unfiltered}
\end{figure}

\noindent Figure~\ref{unfiltered} shows the genre distribution over the
full data set. Unfortunately, we realized too late, that the empty genre
is considered a class, and therefore, that the data is heavily
imbalanced. This influenced most of the experiments we did with the set,
delivering 53$\%$ accuracy for some of our neural networks, which only
predicted the empty genre. The individual chapters will provide more
insight.\\

\begin{figure}[]
	\centering
	\includegraphics[width=1.0\textwidth]{images/genredistfiltered.png}
	\caption{Distribution of the genres over the filtered dataset with no empty
	top genre.}	
	\label{filtered}
\end{figure}

Therefore, we had to filter the songs from the set which were untagged,
as they hinder training not only by delivering high accuracy for the
prediction of a single class, but also, because the songs are probably
from one of the other 16 classes in reality and therefore have similar
features to the remaining songs, but a different tag.
Figure~\ref{filtered} pictures the distribution of the genres after the
tracks with no top genre got filtered out. As you can see, the data is
still imbalanced, with a majority of the songs being either Rock,
Experimental or Electronic music. To improve results in the future, we
should filter out genres with small sample sizes. Due to the deadline,
we were only able to redo some of our tests we already carried out on
the full dataset.