\chapter{Introduction}
\textit{by Patrick Dammann}

\bigskip

The automated analysis of music is an expanding field, mainly because of the increasing popularity of music streaming services. Since these companies usually earn money through monthly subscriptions and can only minimally compete through their product range, they have to satisfy their customers with features and comfort.

One highly demanded feature is a good song recommendation, since it is hard for people to find new music they like, especially in the sheer infinite amount of music most services offer. This is where modern machine learning approaches can shine, since good recommendation needs good analysis of the listeners, as well as the music they are listening to.

Here, genre classification comes into account. While the main results of the successful training of a genre-classification-model might not be very helpful in this terms, finished models can be used to extract features that can give an abstract and numerical interpretation of songs to improve future analysis of the musical tastes of individuals to help them finding new music to listen to.

Aside from capitalistic motivations, this project was also designed to get better insights on how neural networks work on music data, since it is handled very differently from common data like images, texts and even sound data containing speech.

\section{Outlook}

This project will discuss several methods for music genre classification. First, we explain how our dataset is made up, to then directly try an approach that uses convolutional neural networks on the raw sample data. The following approach combines complicated, hand-crafted features with a simple, fully-connected neural network. The last method utilizes frequency-based preprocessing together with a convolutional and recurrent neural network.